\documentclass{article}
\usepackage{amsmath}
\usepackage{amsthm}
\usepackage{amssymb}
\usepackage{xcolor}
\usepackage[UTF8]{ctex}
\title{付强习题解答}
\author{}
\date{}
\begin{document}
	
	\maketitle
	
\section{}
\subsection*{(a)}

\textbf{Lemma}: $ f $ is a continuously differentiable function on $\mathbb{R}^n $. Then for any $ \mathbf{x}, \mathbf{y} \in \mathbb{R}^n$, we have
$$
f(\mathbf{y}) \leq f(\mathbf{x}) + \nabla f(\mathbf{x})^T (\mathbf{y} - \mathbf{x}) + \frac{L}{2} \|\mathbf{y} - \mathbf{x}\|_2^2
$$

\textbf{Proof}: Let $ g(t) = f(\mathbf{x} + t(\mathbf{y} - \mathbf{x})) $. Then $ g^\prime(t) = \langle \nabla f(\mathbf{x} + t(\mathbf{y} - \mathbf{x})), \mathbf{y} - \mathbf{x} \rangle $. By the Newton - Leibniz formula, we can obtain
$$
f(\mathbf{y}) - f(\mathbf{x}) = \int_0^1 \langle \nabla f(\mathbf{x} + t(\mathbf{y} - \mathbf{x})), \mathbf{y} - \mathbf{x} \rangle dt
$$

The above equation is equivalent to
$$
f(\mathbf{y}) - f(\mathbf{x}) = \langle \nabla f(\mathbf{x}), \mathbf{y} - \mathbf{x} \rangle + \int_0^1 \langle \nabla f(\mathbf{x} + t(\mathbf{y} - \mathbf{x})) - \nabla f(\mathbf{x}), \mathbf{y} - \mathbf{x} \rangle dt
$$

From the Lipschitz continuous and  Cauchy - Schwarz inequality, we can get

\begin{align*}
	|f(\mathbf{y}) - f(\mathbf{x}) - \langle \nabla f(\mathbf{x}), \mathbf{y} - \mathbf{x} \rangle| &= \left| \int_0^1 \langle \nabla f(\mathbf{x} + t(\mathbf{y} - \mathbf{x})) - \nabla f(\mathbf{x}), \mathbf{y} - \mathbf{x} \rangle dt \right| \\
	&\leq \int_0^1 \left| \langle \nabla f(\mathbf{x} + t(\mathbf{y} - \mathbf{x})) - \nabla f(\mathbf{x}), \mathbf{y} - \mathbf{x} \rangle \right| dt \\
	&\leq \int_0^1 \|\nabla f(\mathbf{x} + t(\mathbf{y} - \mathbf{x})) - \nabla f(\mathbf{x})\|_2 \cdot \|\mathbf{y} - \mathbf{x}\|_2 dt \\
	&\leq \int_0^1 tL\|\mathbf{y} - \mathbf{x}\|_2^2 dt \\
	&= \frac{L}{2} \|\mathbf{y} - \mathbf{x}\|_2^2
\end{align*}

Therefore
$$
f(\mathbf{y}) \leq f(\mathbf{x}) + \nabla f(\mathbf{x})^T (\mathbf{y} - \mathbf{x}) + \frac{L}{2} \|\mathbf{y} - \mathbf{x}\|_2^2  \quad \qed
$$

Fix $ \mathbf{x} \in \mathbb{R}^n $. Define the function
$$
q(\mathbf{y})=f(\mathbf{x}) + \nabla f(\mathbf{x})^T(\mathbf{y} - \mathbf{x})+\frac{L}{2}\|\mathbf{y} - \mathbf{x}\|_2^2
$$

From the Lemma, $ f\mathbf{(y})\leq q(\mathbf{y}) $ holds for all $ \mathbf{y} \in \mathbb{R}^n $. Therefore,
$$
\inf_{\mathbf{y} \in \mathbb{R}^n}f(\mathbf{y})\leq\inf_{\mathbf{y} \in \mathbb{R}^n}q(\mathbf{y})
$$

Let $ \mathbf{d} = \mathbf{y} - \mathbf{x} $. Then
$$
q(\mathbf{y})=f(\mathbf{x})+\nabla f(\mathbf{x})^T \mathbf{d}+\frac{L}{2}\|\mathbf{d}\|_2^2
$$

The gradient of this quadratic function with respect to $ \mathbf{d} $ is $ \nabla_q = \nabla f(\mathbf{x})+L\mathbf{d} $. Set this gradient to zero:
$$
\nabla f(\mathbf{x})+L\mathbf{d} = 0\implies \mathbf{d} = -\frac{1}{L}\nabla f(\mathbf{x})
$$

Therefore, the minimum point is
$$
\mathbf{y}^*=\mathbf{x} + \mathbf{d}=\mathbf{x}-\frac{1}{L}\nabla f(\mathbf{x})
$$

Substitute into $ q(y) $ to get the minimum value:
$$
q(\mathbf{y}^*)=f(\mathbf{x})+\nabla f(\mathbf{x})^T\left(-\frac{1}{L}\nabla f(\mathbf{x})\right)+\frac{L}{2}\left\|-\frac{1}{L}\nabla f(\mathbf{x})\right\|_2^2=f(\mathbf{x})-\frac{1}{2L}\|\nabla f(\mathbf{x})\|_2^2
$$

That is,
$$
\inf_{\mathbf{y} \in \mathbb{R}^n}q(\mathbf{y})=q(\mathbf{y}^*)=f(\mathbf{x})-\frac{1}{2L}\|\nabla f(\mathbf{x})\|_2^2
$$

From $ f(\mathbf{y})\leq q(\mathbf{y}) $ and $ \inf_{\mathbf{y}}f(\mathbf{y})\leq\inf_{\mathbf{y}}q(\mathbf{y}) $, we get
$$
\inf_{\mathbf{y} \in \mathbb{R}^n}f(\mathbf{y})\leq f(\mathbf{x})-\frac{1}{2L}\|\nabla f(\mathbf{x})\|_2^2
$$

This inequality holds for any $ \mathbf{x} \in \mathbb{R}^n $. 

\subsection*{(b)}
Fix $ \mathbf{x} \in \mathbb{R}^n $. Define the function
$$
p(\mathbf{y})=f(\mathbf{y})-\nabla f(\mathbf{x})^T \mathbf{y}
$$

Since $ f $ is a convex function, after subtracting a linear term, $ p $ is still a convex function.
Because $ \nabla f $ is $ L $-Lipschitz continuous, $ \nabla p $ is also $ L $-Lipschitz continuous.
\\
At the point $ \mathbf{y} = \mathbf{x} $, calculate the gradient:
$$
\nabla p(\mathbf{x})=0
$$

 Since $ p $ is a convex function and $ \nabla p(\mathbf{x}) = 0 $, $ p $ attains the global minimum at $ \mathbf{x} $, that is
$$
\inf_{\mathbf{z} \in \mathbb{R}^n}p(\mathbf{z})=p(\mathbf{x})
$$

Problem (a) shows that 
$$
\inf_{\mathbf{z} \in \mathbb{R}^n}p(\mathbf{z})\leq p(\mathbf{y})-\frac{1}{2L}\|\nabla p(\mathbf{y})\|_2^2,\quad \forall \mathbf{y} \in \mathbb{R}^n
$$
	
Substitute $ \inf_{\mathbf{z} \in \mathbb{R}^n}p(\mathbf{z})=p(\mathbf{x}) $, we get
$$
p(\mathbf{x})\leq p(\mathbf{y})-\frac{1}{2L}\|\nabla p(\mathbf{y})\|_2^2
$$

Equivalently,
$$
p(\mathbf{x})-g(\mathbf{y})\leq -\frac{1}{2L}\|\nabla g(\mathbf{y})\|_2^2
$$

Then, we inspect $ p(\mathbf{x}) - p(\mathbf{y}) $ and $ \|\nabla p(\mathbf{y})\|_2^2 $
\begin{align*}
	p(\mathbf{x}) - p(\mathbf{y}) &= \left[ f(\mathbf{x}) - \nabla f(\mathbf{x})^T x \right] - \left[ f(\mathbf{y}) - \nabla f(\mathbf{x})^T \mathbf{y} \right] \\
	&= f(\mathbf{x}) - f(\mathbf{y}) - \nabla f(\mathbf{x})^T (\mathbf{x} - \mathbf{y})
\end{align*}
$$
\|\nabla p\mathbf{(y})\|_2^2 = \|\nabla f(\mathbf{y}) - \nabla f(\mathbf{x})\|_2^2 = \|\nabla f(\mathbf{x}) - \nabla f(\mathbf{y})\|_2^2
$$

Summarizing
$$
{f(\mathbf{x}) - f(\mathbf{y}) - \nabla f(\mathbf{x})^T (\mathbf{x} - \mathbf{y}) \leq -\frac{1}{2L} \|\nabla f(\mathbf{x}) - \nabla f(\mathbf{y})\|_2^2}
$$

\section{}
\subsection*{(a) $\Rightarrow$ (b)}

Assume that $ A $ is not positive semi - definite. Then there exists a vector $ \mathbf{v} \in \mathbb{R}^n $, $ \mathbf{v} \neq \mathbf{0} $, such that $ \mathbf{v}^TA\mathbf{v}< 0 $. Consider $ \mathbf{x}_t = t\mathbf{v} $ where $ t \in \mathbb{R} $. Substitute it into $ q $:
$$
q(\mathbf{x}_t)=q(t\mathbf{v})=\frac{1}{2}(t\mathbf{v})^TA(t\mathbf{v})-B^T(t\mathbf{v})=\frac{1}{2}t^2(\mathbf{v}^TA\mathbf{v})-t(B^T\mathbf{v})
$$

Since $ \mathbf{v}^TA\mathbf{v}< 0 $, when $ t \to \infty $, the quadratic term $ \frac{1}{2}t^2(\mathbf{v}^TA\mathbf{v})\to -\infty $, and another term $ -t(B^T\mathbf{v}) $, so $ q(t\mathbf{v})\to -\infty $. This contradicts the fact that $q$ is bounded below. Therefore, $ A $ must be positive semi - definite, namely, $ A \succeq 0 $.

Since $ A $ is symmetric, there is an orthogonal decomposition $ \mathbb{R}^n=\text{range}(A)\oplus \text{ker}(A) $. Let $ B=B_r+B_n $, where $ B_r \in \text{range}(A) $, $ B_n \in \text{ker}(A) $, and $ B_r^TB_n = 0 $. Assume $ B_n\neq \mathbf{0} $. Consider $ \mathbf{x}_t = tB_n $ where $ t \in \mathbb{R} $. Substitute it into $q $:
$$
q(\mathbf{x}_t)=q(tB_n)=\frac{1}{2}(tB_n)^TA(tB_n)-B^T(tB_n)=\frac{1}{2}t^2(B_n^TAB_n)-t(B^TB_n)
$$

Because $ B_n \in \text{ker}(A) $, we have $ AB_n=\mathbf{0} $, so $ B_n^TAB_n = 0 $. Further:
$$
B^TB_n=(B_r + B_n)^TB_n=B_r^TB_n+B_n^TB_n = 0+\|B_n\|^2>0\quad (\text{since } B_n\neq \mathbf{0})
$$

Then:
$$
q(tB_n)=-t\|B_n\|^2
$$

When $ t \to \infty $, $ q(tB_n)\to -\infty $, which contradicts the fact that $ q $ is bounded below. Therefore, $ B_n=\mathbf{0} $, that is, $ B \in \text{range}(A)$.

\subsection*{(b) $\Rightarrow$ (c)}

Assume that $ A \succeq 0 $ and $ B \in \text{range}(A) $. Then there exists $ \mathbf{x}^* \in \mathbb{R}^n $ such that $ A\mathbf{x}^* = B $. Calculate the gradient:
$$
\nabla q(\mathbf{x}) = A\mathbf{x} - B
$$

At $ \mathbf{x}^* $:
$$
\nabla q(\mathbf{x}^*) = A\mathbf{x}^* - B = \mathbf{0}
$$

The Hessian matrix is $ \nabla^2 q(\mathbf{x}) = A $, and since $ A \succeq 0 $, the Hessian at $ \mathbf{x}^* $ is positive semi - definite. Therefore, $ \mathbf{x}^* $ is a local minimum point of $ q $. So $ q $ has a local minimum.

\subsection*{(c) $\Rightarrow$ (d)}

Assume that $ \mathbf{x}^* $ is a local minimum point. At $ \mathbf{x}^* $:
The gradient is zero: $ \nabla q(\mathbf{x}^*) = A\mathbf{x}^* - B = \mathbf{0} $, so $ A\mathbf{x}^* = B $, that is, $ B \in \text{range}(A) $.
The Hessian matrix $ A $ is positive semi - definite is a local minimum point), so, $ A \succeq 0 $.

From $ A \succeq 0 $ and $ A\mathbf{x}^* = B $, consider the function values:
$$
q(\mathbf{x}) - q(\mathbf{x}^*) = \left( \frac{1}{2}\mathbf{x}^TA\mathbf{x} - B^T\mathbf{x} \right)-\left
(\frac{1}{2}(\mathbf{x}^*)^TA\mathbf{x}^* - B^T\mathbf{x}^* \right)
$$

Substitute $ B = A\mathbf{x}^* $:
$$
B^T\mathbf{x} = (A\mathbf{x}^*)^T\mathbf{x} = \mathbf{x}^TA\mathbf{x}^*, \quad B^T\mathbf{x}^* = (A\mathbf{x}^*)^T\mathbf{x}^* = (\mathbf{x}^*)^TA\mathbf{x}^*
$$

Therefore:
$$
q(\mathbf{x}) - q(\mathbf{x}^*) = \frac{1}{2}\mathbf{x}^TA\mathbf{x} - \mathbf{x}^TA\mathbf{x}^* - \frac{1}{2}(\mathbf{x}^*)^TA\mathbf{x}^* + (\mathbf{x}^*)^TA\mathbf{x}^* = \frac{1}{2}\mathbf{x}^TA\mathbf{x} - \mathbf{x}^TA\mathbf{x}^* + \frac{1}{2}(\mathbf{x}^*)^TA\mathbf{x}^*
$$

Thus:
$$
q(\mathbf{x}) - q(\mathbf{x}^*) = \frac{1}{2}(\mathbf{x} - \mathbf{x}^*)^TA(\mathbf{x} - \mathbf{x}^*)
$$

Because $ A \succeq 0 $, we have $ (\mathbf{x} - \mathbf{x}^*)^TA(\mathbf{x} - \mathbf{x}^*) \geq 0 $. So $ q(\mathbf{x}) - q(\mathbf{x}^*) \geq 0 $, that is, $ q(\mathbf{x}) \geq q(\mathbf{x}^*) $ holds for all $ \mathbf{x} \in \mathbb{R}^n $. Therefore, $ \mathbf{x}^* $ is a global minimum point, that is, $ q $ has a global minimum. So $ (c) \Rightarrow (d) $.

\subsection*{(d) $\Rightarrow$ (a)}
Easy to prove.

\section{}
For $\forall t < t_0$, we can get $\mathcal{L}(t)\subset \mathcal{L}(t_0)$. Because of the Boundedness of $\mathcal{L}(t_0)$, $\mathcal{L}(t)$ is bounded. Therefore, we We just need to prove $\forall t > t_0$, $\mathcal{L}(t)$ is bounded.
 
Assume that $\exists t_1 > t_0$, such that $\mathcal{L}(t_1)$ is unbounded, namely $\exists \{\mathbf{x}_k \} \subset \mathcal{L}(t_1)$, such that, $\| \mathbf{x}_k\|_2 \rightarrow \infty$. Consider that $\mathbf{x}_0 \in \mathcal{L}(t_0)$ and $\mathbf{x}_k$, let 
 
$$\mathbf{y}=\lambda\mathbf{x}_0+(1 - \lambda)\mathbf{x}_k , \quad \lambda\in(0,1) 
$$

According to convexity:
$$
f(\mathbf{y}) \leq\lambda f(\mathbf{x}_0) +(1-\lambda)f(\mathbf{x}_k)\leq\lambda t_0+(1 - \lambda)t_1
$$

Let $ \lambda = \frac{k-1}{k} $, $ f(\mathbf{y}) \leq \frac{k-1}{k}t_0+ \frac{t_1}{k}$. When $k \rightarrow \infty $, $f(\mathbf{y}) \leq t_0$, namely $\mathbf{y}$ is bounded. But when $k \rightarrow \infty $, $ \|\mathbf{y}\|_2 \rightarrow \| \mathbf{x}_k\|_2 \rightarrow \infty $, this contradicts $\mathbf{y}$ is bounded. Therefore, $\mathcal{L}(t)$ is bounded, when $t_1 > t_0$.


\section{}
 
\textbf{Lemma}: Subgradients on a compact set must be bounded.

\textbf{Proof}:

Take $ \delta > 0 $, and define $ K_\delta = \{ \mathbf{y} : \text{d}(\mathbf{y}, K) =inf_{\mathbf{z}\in K}\|\mathbf{y}-\mathbf{z}\| \leq \delta \} $.  Since $ K $ is compact, $ K_\delta $ is compact.

Since $ f $ is convex function, $ f $ is continuous on the compact set $ K_\delta $, so there $\exists$:  
$$
M_\delta = \sup_{\mathbf{z} \in K_\delta} f(\mathbf{z}), \quad m_\delta = \inf_{\mathbf{z} \in K_\delta} f(\mathbf{z}), \quad \omega = M_\delta - m_\delta < \infty
$$  

For any $ \mathbf{x} \in K $ and $ g \in \partial f(\mathbf{x}) $, let $ d = g / \|g\| $ (if $ g \neq 0 $) and $ \mathbf{y} = \mathbf{x} + \delta d \in \overline{B}(\mathbf{x}, \delta) \subset K_\delta $. By the definition of subgradients:  
$$
f(\mathbf{y}) \geq f(\mathbf{x}) + g^\top (\mathbf{y} - \mathbf{x}) = f(\mathbf{x}) + \delta \|g\|
$$  

From $ f(\mathbf{y}) \leq M_\delta $ and $ f(\mathbf{x}) \geq m_\delta $, we can get:  

$$\delta \|g\| \leq f(\mathbf{y}) - f(\mathbf{x}) \leq \omega \implies \|g\| \leq \frac{\omega}{\delta} \quad \qed
$$  

For any $ \mathbf{x}, \mathbf{y} \in K $, consider $(1 - t)\mathbf{x} + t\mathbf{y} (t \in [0, 1]) $. By convexity,  there $\exists  g_t \in \partial f((1 - t)\mathbf{x} + t\mathbf{y}) $ such that:
$$f(\mathbf{y}) - f(\mathbf{x}) = \int_{0}^{1} \frac{d}{dt}f((1 - t)\mathbf{x} + t\mathbf{y}) dt = \int_{0}^{1} g_t^\top (\mathbf{y} - \mathbf{x}) dt
$$
$$
|f(\mathbf{y}) - f(\mathbf{x})| \leq \int_{0}^{1} |g_t^\top (\mathbf{y} - \mathbf{x})| dt \leq \int_{0}^{1} \|g_t\| \cdot \|\mathbf{y} - \mathbf{x}\| dt
$$
From the lemma we know the subgradient on $ K $ set is bounded. And
$$
|f(\mathbf{y}) - f(\mathbf{x})| \leq \frac{\omega}{\delta} \|\mathbf{y} - \mathbf{x}\|
$$
 
Therefore, the convex function $ f $ is $L$- Lipschitz continuous on the compact set $ K $, where $L=\frac{\omega}{\delta}$.
 
\section{}
 
We need prove that $ \|\ \mathbf{x} - t\nabla f(\mathbf{x}) - \mathbf{x}^*\|_2 \leq \|\mathbf{x} - \mathbf{x}^*\|_2 $.
 
That is
$$
\|\ \mathbf{x} - t\nabla f(\mathbf{x}) - \mathbf{x}^*\|_2^2 \leq \|\mathbf{x} - \mathbf{x}^*\|_2^2
$$
 
Consider that $\|\ \mathbf{x} - t\nabla f(\mathbf{x}) - \mathbf{x}^*\|_2^2$
$$
\|\ \mathbf{x} - t\nabla f(\mathbf{x}) - \mathbf{x}^*\|_2^2 = \| \mathbf{x} - \mathbf{x}^*\|_2^2 + t^2 \|\ \nabla f(\mathbf{x}) \|_2^2 -2t \langle\nabla f(\mathbf{x}), \mathbf{x} - \mathbf{x}^* \rangle
$$ 

That is prove
$$
 t^2 \|\ \nabla f(\mathbf{x}) \|_2^2 \leq 2t \langle\nabla f(\mathbf{x}), \mathbf{x} - \mathbf{x}^* \rangle
$$

Becasue of $ t>0 $, collating the above inequation, that is
$$
t \|\ \nabla f(\mathbf{x}) \|_2^2 \leq 2 \langle\nabla f(\mathbf{x}), \mathbf{x} - \mathbf{x}^* \rangle
$$

$f$ is convex function, and $\nabla f(\mathbf{x})$ is $L$-Lipschitz continous. From the question(1), we know that
$$
\frac{1}{L} \|\nabla f(\mathbf{x}) - \nabla f(\mathbf{y})\|_2^2 \leq \langle \nabla f(\mathbf{x})-\nabla f(\mathbf{y}), \mathbf{x} - \mathbf{y} \rangle
$$

Let $\mathbf{y} =\mathbf{x}^* $, that is
$$
\frac{1}{L} \|\nabla f(\mathbf{x})\|_2^2 \leq \langle \nabla f(\mathbf{x}), \mathbf{x} - \mathbf{x}^* \rangle
$$

So
$$
\frac{2}{L} \|\nabla f(\mathbf{x})\|_2^2 \leq 2 \langle \nabla f(\mathbf{x}), \mathbf{x} - \mathbf{x}^* \rangle
$$

We can easily know that, $ \|\ \mathbf{x} - t\nabla f(\mathbf{x}) - \mathbf{x}^*\|_2 \leq \|\mathbf{x} - \mathbf{x}^*\|_2 $. where $t \in (0,\frac{2}{L}) $. If and only if $ \mathbf{x}= \mathbf{x}^*$ the inequation takes an equal sign, otherwise the inequation strictly holds.

\section{}

I don't think such a function exists.

Assume that there exists a convex function $ f $ that is differentiable but not continuously differentiable on an open convex set $ U \subseteq \mathbb{R}^n $. Then there exists a point $ \mathbf{x} \in U $ and a sequence $ \{\mathbf{x}_k\} \subseteq U $ converging to $ \mathbf{x} $ (i.e., $ \mathbf{x}_k \to \mathbf{x} $), but the sequence of gradients $ \{\nabla f(\mathbf{x}_k)\} $ does not converge to $ \nabla f(\mathbf{x}) $. That is:
$$
\nabla f(\mathbf{x}_k) \nrightarrow \nabla f(\mathbf{x}) \quad \text{as} \quad k \to \infty
$$

Since $ U $ is an open set and $ \mathbf{x} \in U $, there exists a neighborhood $ K \subseteq U $ containing $ \mathbf{x} $. Because $ f $ is convex on $ U $, it is Lipschitz continuous on $ K $ . Let the Lipschitz constant be $ L $. If $ f $ is differentiable, then the gradient is bounded on $ K $: for all $ \mathbf{y} \in K $, $ \|\nabla f(\mathbf{y})\| \leq L $.

The sequence $ \{\nabla f(\mathbf{x}_k)\} $ is bounded, so it has a convergent subsequence. Assume the entire sequence converges (otherwise take a subsequence), that is:
$$
\nabla f(\mathbf{x}_k) \to \mathbf{g} \quad \text{as} \quad k \to \infty
$$
where $ \mathbf{g} \neq \nabla f(\mathbf{x}) $.

Since $ f $ is convex and differentiable on $ U $, for any $ \mathbf{y} \in U $, the subgradient inequality holds:
$$
f(\mathbf{y}) \geq f(\mathbf{x}_k) + \langle \nabla f(\mathbf{x}_k), \mathbf{y} - \mathbf{x}_k \rangle
$$

Take the limit as $ k \to \infty $:
$ f(\mathbf{x}_k) \to f(\mathbf{x}) $ (because $ f $ is continuous; a convex function is continuous on an open set).
$ \nabla f(\mathbf{x}_k) \to \mathbf{g} $.
$ \mathbf{x}_k \to \mathbf{x} $, so $ \mathbf{y} - \mathbf{x}_k \to \mathbf{y} - \mathbf{x}$.

Thus:
$$
f(\mathbf{y}) \geq \lim_{k \to \infty} \left[ f(\mathbf{x}_k) + \langle \nabla f(\mathbf{x}_k), \mathbf{y} - \mathbf{x}_k \rangle \right] = f(\mathbf{x}) + \langle \mathbf{g}, \mathbf{y} - \mathbf{x} \rangle
$$

This shows that $ \mathbf{g} $ is a subgradient of $ f $ at $ \mathbf{x} $, i.e., $ \mathbf{g} \in \partial f(\mathbf{x}) $. But  $ \partial f(\mathbf{x}) = \{\nabla f(\mathbf{x})\} $. Therefore, we must have $ \mathbf{g} = \nabla f(\mathbf{x}) $, which contradicts the assumption $ \mathbf{g} \neq \nabla f(\mathbf{x}) $.

\section{}

At $ x^*=0 $, $f$ does not necessarily have a local minimum.

Let $ g(t) = f(t\mathbf{d}) $, $ g'(t) = \langle\nabla f(t\mathbf{d}), \mathbf{d} \rangle$, $g''(t)= \langle\nabla^2  f(t\mathbf{d}), \|\mathbf{d} \|_2^2 \rangle \ $. Since $ t \mapsto f(t\mathbf{d}) $ has a local minimum at $ t^* = 0 $, we can know that
$$
g'(0)= \langle\nabla f(\mathbf{0}), \mathbf{d} \rangle\ =0
$$
$$
g''(0)= \nabla^2  f(\mathbf{0}) \geq 0
$$

Therefore,$ \nabla f(\mathbf{0})=0$, $\nabla^2  f(\mathbf{0}) \geq 0$. this implies that $ \nabla^2 f(\mathbf{0}) $ is positive semi - definite. This means that $ f $ has a local minimum or a saddle point at $ \mathbf{x}^* = 0 $.

\section{}

\( x^* \) is not necessarily a global minimizer. The following is a counterexample.

Consider the function \( f: \mathbb{R}^2 \to \mathbb{R} \)
$$f(x, y) = x^2 + y^2(1 - x)^3$$

This function is twice continuously differentiable.

Calculate the gradient:
$$
\nabla f(x, y) = \left( \frac{\partial f}{\partial x}, \frac{\partial f}{\partial y} \right)
$$
where
$$\frac{\partial f}{\partial x} = 2x - 3y^2(1 - x)^2, \quad \frac{\partial f}{\partial y} = 2y(1 - x)^3
$$

\( \frac{\partial f}{\partial y} = 0 \) gives \( 2y(1 - x)^3 = 0 \), so \( y = 0 \) or \( x = 1 \).\\
If \( x = 1 \), then \( \frac{\partial f}{\partial x} = 2(1) - 3y^2(1 - 1)^2 = 2 \neq 0 \). Thus, \( x = 1 \) does not satisfy the condition that the gradient is zero.\\
If \( y = 0 \), then \( \frac{\partial f}{\partial x} = 2x - 0 = 2x \). Setting this equal to zero gives \( x = 0 \).

Therefore, the unique critical point is \( (x, y) = (0, 0) \).


At \( (0, 0) \), \( f(0, 0) = 0 \).

The Hessian matrix is:
$$
H_f(x, y) = \begin{pmatrix}
	\frac{\partial^2 f}{\partial x^2} & \frac{\partial^2 f}{\partial x \partial y} \\
	\frac{\partial^2 f}{\partial y \partial x} & \frac{\partial^2 f}{\partial y^2}
\end{pmatrix}
$$
where
$$
\frac{\partial^2 f}{\partial x^2} = 2 + 6y^2(1 - x), \quad \frac{\partial^2 f}{\partial y^2} = 2(1 - x)^3, \quad \frac{\partial^2 f}{\partial x \partial y} = -6y(1 - x)^2
$$

At \( (0, 0) \):
$$
H_f(0, 0) = \begin{pmatrix}
	2 & 0 \\
	0 & 2
\end{pmatrix}
$$

The eigenvalues are \( 2 > 0 \), so the matrix is positive definite. Therefore, \( (0, 0) \) is a local minimizer.

Take the point \( (2, 3) \):
$$
f(2, 3) = 2^2 + 3^2(1 - 2)^3 = 4 + 9 \cdot (-1) = 4 - 9 = -5 < 0 = f(0, 0)
$$

Thus, \( f(2, 3) < f(0, 0) \), so \( (0, 0) \) is not a global minimizer.

\( x^* \) is not necessarily a global minimizer.

\section{}

Since $ \mathbb{P}(X_k = 1)\geq p $ and the goal is to find an upper bound for $ \mathbb{P}(S_n\leq tn) $ (where $ t\leq p $), consider the case when $ \mathbb{P}(X_k = 1)=p $ for all $ k $. In this case, the probability $ \mathbb{P}(S_n\leq tn) $ reaches the maximum. Therefore, to find the upper bound, we can assume that each $ X_k $ is an independent Bernoulli random variable with parameter $ p $, that is, $ S_n\sim \text{Binomial}(n,p) $.

For the lower tail of a binomial distribution, the standard Chernoff bound states: Let $ \mu=\mathbb{E}[S_n]=np $. For $ \delta\in[0,1] $, we have
$$
\mathbb{P}(S_n\leq(1 - \delta)\mu)\leq\exp\left(-\frac{\delta^2\mu}{2}\right)
$$

Let $ (1 - \delta)\mu = tn $. Substitute $ \mu = np $:
$$
(1 - \delta)np=tn\implies 1 - \delta=\frac{t}{p}\implies\delta = 1-\frac{t}{p}=\frac{p - t}{p}
$$

Substitute into the Chernoff bound:
$$
\mathbb{P}(S_n\leq tn)\leq\exp\left(-\frac{\left(\frac{p - t}{p}\right)^2\cdot(np)}{2}\right)=\exp\left(-\frac{(p - t)^2\cdot np}{2p^2}\right)=\exp\left(-\frac{(p - t)^2n}{2p}\right)
$$

In the general case, $ \mathbb{P}(X_k = 1)=p_k\geq p $. To prove the upper bound, we use the general form of the Chernoff bound: For any $ \lambda\leq0 $, we have
$$
\mathbb{P}(S_n\leq tn)\leq e^{-\lambda tn}\prod_{k = 1}^n\mathbb{E}[e^{\lambda X_k}]
$$

For each $ k $, the moment - generating function $ \mathbb{E}[e^{\lambda X_k}]=1 - p_k + p_ke^{\lambda} $. Consider the function $ h(p)=1 - p+pe^{\lambda} $. Its derivative is
$$
\frac{\partial h}{\partial p}=- 1+e^{\lambda}
$$

Since $ \lambda\leq0 $, $ e^{\lambda}\leq1 $, so $ \frac{\partial h}{\partial p}\leq0 $, that is, $ h(p) $ is non - increasing in $ p $. Therefore, when $ p_k\geq p $,
$$
\mathbb{E}[e^{\lambda X_k}]=h(p_k)\leq h(p)=1 - p + pe^{\lambda}
$$

Thus,
$$
\prod_{k = 1}^n\mathbb{E}[e^{\lambda X_k}]\leq(1 - p + pe^{\lambda})^n
$$

So,
$$
\mathbb{P}(S_n\leq tn)\leq e^{-\lambda tn}(1 - p + pe^{\lambda})^n
$$

This is the same as in the case of independent and identically distributed Bernoulli($ p $) random variables. By choosing $ \lambda $, we can obtain the same upper bound.

\section{}
\textcolor{red}{$A \in \mathbb{R}^{n}$ should be $A \in \mathbb{R}^{n\times n}$ in the question}

$\rho$ is not a consistent matrix norm on $\mathbb{R}^{n\times n}$, $\rho$ satisfies (a), and violates (b), (c), (d).

\subsection*{(a)}

 If $ \lambda $ is an eigenvalue of $ A $, then $ \alpha\lambda $ is an eigenvalue of $ \alpha A $. Thus, $ \rho(\alpha A)=\max|\alpha\lambda| = |\alpha|\max|\lambda|=|\alpha|\rho(A) $.

\subsection*{(b)}

Let
$$
A=\begin{pmatrix}0&1\\0&0\end{pmatrix}, \quad B=\begin{pmatrix}0&0\\1&0\end{pmatrix}
$$

Then $ \rho(A) = 0 $ , $ \rho(B)=0 $ , but
$$
A + B=\begin{pmatrix}0&1\\1&0\end{pmatrix}
$$

The eigenvalues are $ 1, - 1 $, so $ \rho(A + B)=1 $.

So $ 1=\rho(A + B)>\rho(A)+\rho(B)= 0 $.

\subsection*{(c)}

Let
$$
A=\begin{pmatrix}0&1\\0&0\end{pmatrix}
$$

The eigenvalues are $ 0,0 $, so $ \rho(A)=0 $, but $ A\neq0 $.

\subsection*{(d)}

Let
$$
A=\begin{pmatrix}0&1\\0&0\end{pmatrix}, \quad B=\begin{pmatrix}0&0\\1&0\end{pmatrix}
$$

Then
$$
AB=\begin{pmatrix}0&1\\0&0\end{pmatrix}\begin{pmatrix}0&0\\1&0\end{pmatrix}=\begin{pmatrix}1&0\\0&0\end{pmatrix}
$$

The eigenvalues are $ 1,0 $, so $ \rho(AB)=1 $.

We have $ \rho(A)=0 $, $ \rho(B)=0 $, so $ \rho(A)\rho(B)=0 $, but $ 1=\rho(AB)>\rho(A)\rho(B)=0 $.

\section{}

\subsection*{(a)}
$$
\| \mathbf{x}+\mathbf{y}\|_p^p \leq \| \mathbf{x}\|_p^p +\| \mathbf{y}\|_p^p \Leftrightarrow \sum_{i=1}^{n} (x_i+y_i)^p \leq \sum_{i=1}^{n} x_i^p+\sum_{i=1}^{n}y_i^p
$$

we only need to prove that, for every $x_i, y_i > 0$, $(x_i+y_i)^p \leq  x_i^p+y_i^p$

Consider $f(t)=t^p ( t \geq 0,p\in (0,1])$, $f''(t)=p(p-1)t^(p-2)$, easy to know $f''(t) \geq 0$, so $f(t)$ is a concave function. Therefore, we can know $x_i, y_i > 0$, $(x_i+y_i)^p \leq  x_i^p+y_i^p$

\subsection*{(b)}   

From the (a), we have $\| \mathbf{x}+\mathbf{y}\|_p^p \leq \| \mathbf{x}\|_p^p +\| \mathbf{y}\|_p^p$, so
\begin{align*}
	\| \mathbf{x}+\mathbf{y}\|_p 
	&\leq (\| \mathbf{x}\|_p^p +\| \mathbf{y}\|_p^p)^\frac{1}{p} \\
	&= (\| \mathbf{x}\|_p +\| \mathbf{y}\|_p)[(\frac{\mathbf{x}_p}{\| \mathbf{x}\|_p +\| \mathbf{y}\|_p})^p+(\frac{\mathbf{y}_p}{\| \mathbf{x}\|_p +\| \mathbf{y}\|_p})^p]^\frac{1}{p} \\
	&= (\| \mathbf{x}\|_p +\| \mathbf{y}\|_p)[(\frac{\mathbf{x}_p}{\| \mathbf{x}\|_p +\| \mathbf{y}\|_p})^p+(1-\frac{\mathbf{x}_p}{\| \mathbf{x}\|_p +\| \mathbf{y}\|_p})^p]^\frac{1}{p}
\end{align*}

Let $g(t)=t^p+(1-t)^p$, where $0<p<1, t\in(0,1)$

$g'(t)=pt^{p-1}+(1-t)^{p-1} $ When t=0.5, $g'(t)=0$, easy to know $g_{max}=g(0.5)=2^{1-p}$

$\frac{\mathbf{x}_p}{\| \mathbf{x}\|_p +\| \mathbf{y}\|_p} \in (0,1)$, so $[(\frac{\mathbf{x}_p}{\| \mathbf{x}\|_p +\| \mathbf{y}\|_p})^p+(1-\frac{\mathbf{x}_p}{\| \mathbf{x}\|_p +\| \mathbf{y}\|_p})^p]_{max}=2^{1-p}$

From the above inequality, we can have:
$$
\| \mathbf{x}+\mathbf{y}\|_p \leq (2^{p-1})^\frac{1}{p}(\| \mathbf{x}\|_p +\| \mathbf{y}\|_p) =2^{\frac{1}{p}-1}(\| \mathbf{x}\|_p +\| \mathbf{y}\|_p)
$$

\subsection*{(c)} 

$$
\| \mathbf{x}+\mathbf{y}\|_p \geq \| \mathbf{x}\|_p +\| \mathbf{y}\|_p \Leftrightarrow \frac{\| \mathbf{x}+\mathbf{y}\|_p}{\|\mathbf{x}\|_p +\| \mathbf{y}\|_p} \geq 1
$$

$$
\frac{\| \mathbf{x}+\mathbf{y}\|_p}{\|\mathbf{x}\|_p +\| \mathbf{y}\|_p}=[\frac{\sum (x_i+y_i)^p}{(\| \mathbf{x}\|_p +\| \mathbf{y}\|_p)^p}]^{\frac{1}{p}}
$$

For function $g(u)=u^p$, $p\in (0,1)$, $g$ is the concave function, so we can have

$$
[\frac{\sum (x_i+y_i)^p}{(\| \mathbf{x}\|_p +\| \mathbf{y}\|_p)^p}]^{\frac{1}{p}} \geq [\frac{\sum (x_i^p+y_i^p)}{(\| \mathbf{x}\|_p +\| \mathbf{y}\|_p)^p}]^{\frac{1}{p}} = \frac{(\| \mathbf{x}\|_p^p +\| \mathbf{y}\|_p^p)^{\frac{1}{p}}}{(\| \mathbf{x}\|_p +\| \mathbf{y}\|_p)} \geq 1
$$

\subsection*{(d)} 
We need to prove that for $ q > p > 0 $, $ \|\mathbf{x}\|_q \leq \|\mathbf{x}\|_p $. If $ \mathbf{x} = \mathbf{0} $, then all norms are 0, and the statement holds. Let $ \mathbf{x} \neq \mathbf{0} $. Set $ \|\mathbf{x}\|_p = 1 $, then $ \sum |x_i|^p = 1 $. We need to prove $ \|\mathbf{x}\|_q \leq 1 $, that is: $\left( \sum |x_i|^q \right)^{1/q} \leq 1$

Let $ y_i = |x_i|^p \geq 0 $, then $ \sum y_i = 1 $, and:

$$
\|\mathbf{x}\|_q = \left( \sum |x_i|^q \right)^{1/q} = \left( \sum (|x_i|^p)^{q/p} \right)^{1/q} = \left( \sum y_i^{q/p} \right)^{1/q}
$$

Let $ r = q/p > 1 $, then $ \|\mathbf{x}\|_q = \left( \sum y_i^r \right)^{1/q} $. Since $ \sum y_i = 1 $ and $ y_i \geq 0 $, we have $ y_i \leq 1 $. Since $ r > 1 $ and $ y_i \in [0, 1] $, we have $ y_i^r \leq y_i $. Thus:

$$
\sum y_i^r \leq \sum y_i = 1
$$

Therefore:

$$
\left( \sum y_i^r \right)^{1/q} \leq (1)^{1/q} = 1
$$

That is, $ \|\mathbf{x}\|_q \leq 1 = \|\mathbf{x}\|_p $. Equality holds when $ \mathbf{x} $ has only one non - zero component.

\subsection*{(e)} 
\textcolor{red}{Failed to prove it}

$$
log\||\mathbf{x}\|_p=\frac{log(\sum x_i^p)}{p}
$$

Let $h(p)= log(\sum x_i^p) $, $g(p)=\frac{h(p)}{p}$

$$
h'(p)=\frac{\sum x_i^plogx_i}{\sum x_i^p}, \quad h''(p)= \frac{(\sum x_i^p(logx_i)^2)(\sum x_i^p)-(\sum x_i^plogx_i)^2}{(\sum x_i^p)^2} \geq 0
$$

$h$ is a convex function.
$$
g'(t)=\frac{ph'(p)-h(p)}{p^2}, \quad g''(t)=\frac{p^2 h''(p)-2ph'(p)+2h(p)}{p^3}
$$

Since $p>0$, that is $p^3>0$, we only need to prove $p^2 h''(p)-2ph'(p)+2h(p) \geq 0$

\subsection*{(f)} 
$\|A\|_p$ is neither monotonically increasing nor monotonically decreasing for $p>0$.

Consider the matrix $ A=\begin{pmatrix}1&1\\0&0\end{pmatrix} $.

When $ p = 1 $:
$$
\|\mathbf{x}\|_{1}=\left(|x_1|^{1}+|x_2|^{1}\right)^2
$$

Easy to konw that $ \|A\|_{1}=1$

When $ p = 2 $:
$$
\|\mathbf{x}\|_2=\left(|x_1|^2+|x_2|^2\right)^{1/2}=1=x_1^2+x_2^2
$$

$ \|A\mathbf{x}\|_2 = |x_1 + x_2| $. Let $L(x_1,x_2,\lambda)=|x_1+x_2|-\lambda(x_1^2+x_2^2)$.
$$
\frac{\partial L}{\partial x_1}=|-2\lambda x_1|=0
$$
$$
\frac{\partial L}{\partial x_2}=|-2\lambda x_2|=0
$$
$$
\frac{\partial L}{\partial \lambda}=|-2\lambda x_2|=0
$$

We can know that $|x_1 + x_2|$ achieve the maximum value, when $x_1=x_2$, that is,
 $ |x_1 + x_2|\leq\sqrt{2}\|\mathbf{x}\|_2=\sqrt{2} $. Therefore, $ \|A\|_2=\sqrt{2}$.

$ \|A\|_{1}=1<\sqrt{2}=\|A\|_2 $, that is, when $ p $ increases from 1 to 2, $ \|A\|_p $ increases.

Consider the matrix $ A=\begin{pmatrix}1&1\\1&1\end{pmatrix} $.

When $ p = 0.5 $:
$$
\|A\mathbf{x}\|_{0.5}=\left(|x_1 + x_2|^{0.5}+|x_1 + x_2|^{0.5}\right)^2=(2|x_1 + x_2|^{0.5})^2 = 4|x_1 + x_2|
$$

$ \|\mathbf{x}\|_{0.5}=\left(|x_1|^{0.5}+|x_2|^{0.5}\right)^2 = 1 $, that is, $ |x_1|^{0.5}+|x_2|^{0.5}=1 $. Let $ a = |x_1|^{0.5}, b = |x_2|^{0.5} $, then $ a + b = 1 $, $ a,b\geq0 $. Then $ \|A\mathbf{x}\|_{0.5}=4|x_1 + x_2|\leq4(|x_1|+|x_2|)=4(a^2 + b^2) $. Since $ a + b = 1 $, $a^2 + b^2=(a + b)^2-2ab = 1 - 2ab $. The maximum value is achieved when $ ab = 0 $. Therefore, $ \|A\|_{0.5}=4 $.

When $ p = 1 $:
$$
\|\mathbf{x}\|_1=|x_1|+|x_2| = 1
$$

$ \|A\mathbf{x}\|_1=|x_1 + x_2|+|x_1 + x_2| = 2|x_1 + x_2|\leq2(|x_1|+|x_2|)=2 $. Therefore, $ \|A\|_1 = 2 $.

$ \|A\|_{0.5}=4>2=\|A\|_1$. That is, when $ p $ increases from 0.5 to 1, $ \|A\|_p $ decreases.

$\|A\|_p$ is neither monotonically increasing nor monotonically decreasing.

\section{\textcolor{red}{Failed to prove it}}


\section{}
\subsection*{(a)} 
\subsubsection*{(1)} 
The eigenvalues of $AB$ are all eigenvalues of $BA$:

$\lambda \neq 0 $ is an eigenvalue of $AB$ and the corresponding eigenvector is $\mathbf{x}\in \mathbb{C}^m$, that is $AB\mathbf{x}=\lambda\mathbf{x}$, so
\begin{align*}
	BAB\mathbf{x}&= B (\lambda\mathbf{x})\\
	BA(B\mathbf{x})&=\lambda(B\mathbf{x})
\end{align*}

It shows that $B\mathbf{x}$ is an eigenvector of $BA$, $\lambda  $ is an eigenvalue of $AB$. If $ B\mathbf{x}=\mathbf{0} $, then the original equation becomes $ AB \mathbf{x}=\lambda\mathbf{x}=\mathbf{0} $. Since $ \lambda\neq0 $, we must have $ \mathbf{x}=\mathbf{0} $, which contradicts the fact that an eigenvector is non - zero. Therefore, $ B\mathbf{x}\neq\mathbf{0} $, that is, $ \lambda $ is a non-zero eigenvalue of $ BA $.

The eigenvalues of $BA$ are all eigenvalues of $AB$:

If $ \lambda\neq0 $ is an eigenvalue of $ BA $, and the corresponding eigenvector is $ \mathbf{y}\in\mathbb{C}^n $, that is, $ BA\mathbf{y}=\lambda\mathbf{y} $.

$$
AB\mathbf{y}=A(\lambda\mathbf{y})\implies AB(A\mathbf{y})=\lambda(A\mathbf{y})$$

Similarly, if $ A\mathbf{y}=\mathbf{0} $, then $ BA\mathbf{y}=\lambda\mathbf{y}=\mathbf{0} $, which leads to $ \mathbf{y}=\mathbf{0} $, a contradiction. Therefore, $ A\mathbf{y}\neq\mathbf{0} $, that is, $ \lambda $ is a non-zero eigenvalue of $ AB $.

\subsubsection*{(2)} 

Consider linear mappings: Let $ A:\mathbb{C}^n\to\mathbb{C}^m $ and $ B:\mathbb{C}^m\to\mathbb{C}^n $ be linear mappings. Then $ AB:\mathbb{C}^m\to\mathbb{C}^m $ and $ BA:\mathbb{C}^n\to\mathbb{C}^n $.

The eigenvalues of $AB$ are all eigenvalues of $BA$:

If $ \lambda\neq0 $ is an eigenvalue of $ AB $, then there $\exists$ a non-zero vector $ \mathbf{x}\in\mathbb{C}^m $ such that $ AB\mathbf{x}=\lambda\mathbf{x} $. $ B:\mathbb{C}^m\to\mathbb{C}^n $, then $ \mathbf{y} = B\mathbf{x}\in\mathbb{C}^n $. We have $ \mathbf{y}\neq0 $ (as mentioned before). Then:
$$
BA\mathbf{y}=BA(B\mathbf{x})=B(AB\mathbf{x})=B(\lambda\mathbf{x})=\lambda B\mathbf{x}=\lambda\mathbf{y}
$$

This shows that $ \mathbf{y} $ is an eigenvector of $ BA $ corresponding to the eigenvalue $ \lambda $.

The eigenvalues of $BA$ are all eigenvalues of $AB$:

If $ \lambda\neq0 $ is an eigenvalue of $ BA $, then there exists a non-zero vector $ \mathbf{z}\in\mathbb{C}^n $ such that $ BA\mathbf{z}=\lambda\mathbf{z} $. $ A:\mathbb{C}^n\to\mathbb{C}^m $, then $ \mathbf{w}=A\mathbf{z}\in\mathbb{C}^m $. We have $ \mathbf{w}\neq0 $ (as mentioned before). Then:
$$
AB\mathbf{w}=AB(A\mathbf{z})=A(BA\mathbf{z})=A(\lambda\mathbf{z})=\lambda A\mathbf{z}=\lambda\mathbf{w}
$$

This shows that $ \mathbf{w} $ is an eigenvector of $ AB $ corresponding to the eigenvalue $ \lambda $.

\subsubsection*{(3)} 

See (1) for details.

\subsection*{(b)} 

Let $ \lambda\neq 0 $ be a common eigenvalue of $ AB $ and $ BA $. Define the eigenspaces:

$ E_{\lambda}(AB)=\{ \mathbf{x}\in\mathbb{C}^m\mid AB\mathbf{x}=\lambda\mathbf{x}\} $, $ E_{\lambda}(BA)=\{ \mathbf{y}\in\mathbb{C}^n\mid BA\mathbf{y}=\lambda\mathbf{y}\} $.

Since $ \lambda\neq 0 $, we can construct linear mappings:

Define $ T: E_{\lambda}(AB)\to E_{\lambda}(BA) $ as $ T(\mathbf{x}) = B\mathbf{x} $, where $ \mathbf{x}\in E_{\lambda}(AB) $.

Define $ S: E_{\lambda}(BA)\to E_{\lambda}(AB) $ as $ S(\mathbf{y}) = A\mathbf{y} $, where $ \mathbf{y}\in E_{\lambda}(BA) $.

Let $ \mathbf{x}_1,\mathbf{x}_2\in E_{\lambda}(AB) $, and $ T(\mathbf{x}_1)=T(\mathbf{x}_2) $, that is, $ B\mathbf{x}_1 = B\mathbf{x}_2 $.

Then $ B(\mathbf{x}_1 - \mathbf{x}_2)=\mathbf{0} $.

Since $ \mathbf{x}_1 - \mathbf{x}_2\in E_{\lambda}(AB) $, we have $ AB(\mathbf{x}_1 - \mathbf{x}_2)=\lambda(\mathbf{x}_1 - \mathbf{x}_2) $.

But $ B(\mathbf{x}_1 - \mathbf{x}_2)=\mathbf{0} $, so:
$$
AB(\mathbf{x}_1 - \mathbf{x}_2)=A(B(\mathbf{x}_1 - \mathbf{x}_2))=A(\mathbf{0})=\mathbf{0}
$$

Thus, $ \lambda(\mathbf{x}_1 - \mathbf{x}_2)=\mathbf{0} $. Since $ \lambda\neq 0 $, we get $ \mathbf{x}_1 - \mathbf{x}_2=\mathbf{0} $, that is, $ \mathbf{x}_1 = \mathbf{x}_2 $.

Therefore, $ T $ is injective.

Similarly, we can prove that $ S $ is injective.

Since $ T: E_{\lambda}(AB)\to E_{\lambda}(BA) $ is injective. Therefore
$$
\dim E_{\lambda}(AB)\leq\dim E_{\lambda}(BA)
$$

Similarly,
$$
\dim E_{\lambda}(BA)\leq\dim E_{\lambda}(AB)
$$

So
$$
\dim E_{\lambda}(AB)=\dim E_{\lambda}(BA)
$$

Therefore, the geometric multiplicities of $ \lambda $ in $ AB $ and $ BA $ are the same.

\subsection*{(c)} 

Denote the characteristic polynomial of $ AB $ as $ f_{AB}(\lambda)=\vert (\lambda I_m - AB)\vert $, and the characteristic polynomial of $ BA $ as $ f_{BA}(\lambda)=\vert(\lambda I_n - BA) \vert $.

Consider the polynomials:
$$
\lambda^n f_{AB}(\lambda) = \lambda^n \vert(\lambda I_m - AB)\vert
$$

and
$$
\lambda^m f_{BA}(\lambda) = \lambda^m \vert(\lambda I_n - BA)\vert
$$

\begin{align*}
	\lambda^n \vert \lambda I_m - AB \vert &= \lambda^n \left\vert \lambda \left( I_m - \frac{1}{\lambda}AB \right) \right\vert = \lambda^n \lambda^m \left\vert I_m - \left( \frac{1}{\lambda}A \right)B \right\vert \\
	&= \lambda^n \lambda^m \left\vert I_n - B \left( \frac{1}{\lambda}A \right) \right\vert = \lambda^m \vert \lambda I_n - BA \vert.
\end{align*}

there is an identity:
$$
\lambda^n \vert(\lambda I_m - AB)\vert = \lambda^m \vert(\lambda I_n - BA)\vert
$$

In the polynomial $ \lambda^n f_{AB}(\lambda) $, the multiplicity of $ \lambda_0 $ as a root is equal to its algebraic multiplicity in $ f_{AB}(\lambda) $. 

Similarly, in the polynomial $ \lambda^m f_{BA}(\lambda) $, the multiplicity of $ \lambda_0 $ as a root is equal to its algebraic multiplicity in $ f_{BA}(\lambda) $.

Since the above identity shows that $ \lambda^n f_{AB}(\lambda) $ and $ \lambda^m f_{BA}(\lambda) $ are the same polynomial, their roots and their multiplicities are completely the same. Therefore, for any non-zero eigenvalue $ \lambda_0 \neq 0 $, its algebraic multiplicities in $ AB $ and $ BA $ are the same.

\section{}

\subsection*{(a)} 

\subsubsection*{(1)} 

$\Rightarrow$

If $ \lambda $ is an eigenvalue of $ A $, then there $\exists$ a non-zero vector $ \mathbf{v}\in\mathbb{C}^n $ such that:
$$
A\mathbf{v}=\lambda\mathbf{v}
$$

For a polynomial $ p(x)=a_0 + a_1x + a_2x^2+\cdots + a_mx^m $, we have:
$$
p(A)=a_0I + a_1A + a_2A^2+\cdots + a_mA^m
$$

Since $ A\mathbf{v}=\lambda\mathbf{v} $, we can obtain that:
$$
A^k\mathbf{v}=\lambda^k\mathbf{v}\quad\text{for all }k\geq0
$$

Therefore:
$$
p(A)\mathbf{v}=(a_0I +\cdots + a_mA^m)\mathbf{v}=a_0\mathbf{v}+\cdots + a_m\lambda^m\mathbf{v}=p(\lambda \mathbf{v})
$$

This shows that $ p(\lambda) $ is an eigenvalue of $ p(A) $.

$\Leftarrow$

If $ p(\lambda) $ is an eigenvalue of $ p(A) $, then there $\exists$ a non-zero vector $ \mathbf{v}\in\mathbb{C}^n $ such that:
$$
p(A)\mathbf{v}=p(\lambda)\mathbf{v}
$$

Since $ p(A)=a_0I + a_1A + a_2A^2+\cdots + a_mA^m $, we have:
$$
(a_0I + a_1A + a_2A^2+\cdots + a_mA^m)\mathbf{v}=p(\lambda)\mathbf{v}
$$

Let $ q(x)=p(x)-p(\lambda) $. Then $ q(\lambda)=0 $, and $ q(A)\mathbf{v}=0 $. Since $ q(x) $ is a polynomial and $ q(\lambda)=0 $, we can write $ q(x)=(x - \lambda)r(x) $, where $ r(x) $ is a polynomial.

Therefore:
$$
q(A)=(A-\lambda I)r(A)
$$

Since $ q(A)\mathbf{v}=0 $, we have:
$$
(A - \lambda I)r(A)\mathbf{v}=0
$$

If $ r(A)\mathbf{v}\neq0 $, then $ A-\lambda I $ must have a non-zero vector $ r(A)\mathbf{v} $ that makes it zero, which means $ \lambda $ is an eigenvalue of $ A $. If $ r(A)\mathbf{v}=0 $, we can continue to apply this process recursively, eventually, we will get that $ \lambda $ is an eigenvalue of $ A $.

\subsubsection*{(2)} 

$\Rightarrow$

If $ \lambda $ is an eigenvalue of $ A $, then there $\exists$ a one - dimensional subspace $ V=\text{span}(\mathbf{v}) $ such that the action of $ A $ on $ V $ is a scaling, i.e., $ A\mathbf{v}=\lambda\mathbf{v} $ (From the a(1)). For the polynomial $ p(A) $, its action on $ V $ is $ p(A)\mathbf{v}=p(\lambda)\mathbf{v} $. Therefore, $ p(\lambda) $ is an eigenvalue of $ p(A) $.

$\Leftarrow$

If $ p(\lambda) $ is an eigenvalue of $ p(A) $, then there exists a one - dimensional subspace $ V = \text{span}(\mathbf{v}) $ such that the action of $ p(A) $ on $ V $ is a scaling, i.e., $ p(A)\mathbf{v}=p(\lambda)\mathbf{v} $. From the a(1), we will get that $ \lambda $ is an eigenvalue of $ A $.

\subsubsection*{(3)} 

See (1) for details.

\subsection*{(b)} 

In a(1) we have already proven that if $ \lambda $ is an eigenvalue of $ A $, $ p(\lambda) $ is an eigenvalue of $ p(A) $. Therefore, we will only consider multiplicities in this question.

If $ \lambda_i $ is an eigenvalue of $ A $ with multiplicity $ k $, then there exist $ k $ linearly independent eigenvectors $ \mathbf{v}_{i1}, \mathbf{v}_{i2}, \ldots, \mathbf{v}_{ik} $ such that:
$$
A\mathbf{v}_{ij} = \lambda_i\mathbf{v}_{ij} \quad \text{for } j = 1, 2, \ldots, k
$$

Through the above process, we can obtain:
$$
p(A)\mathbf{v}_{ij} = p(\lambda_i)\mathbf{v}_{ij} \quad \text{for } j = 1, 2, \ldots, k
$$

This shows that $ p(\lambda_i) $ is an eigenvalue of $ p(A) $ with multiplicity at least $ k $.

Therefore, if $ \lambda_1, \lambda_2, \ldots, \lambda_n $ are the eigenvalues of $ A $ (with repeated roots counted by their multiplicities), then $ p(\lambda_1), p(\lambda_2), \ldots, p(\lambda_n) $ are the eigenvalues of $ p(A) $ (multiple eigenvalues counted with multiplicity).

\section{}

\subsection*{(a)} 

Let $ A = (1 - x)I + xJ $, where $ I $ is the identity matrix and $ J $ is the all - one matrix.

The rank of the all - one matrix $ J $ is 1. Its non - zero eigenvalue is $ n $ (with algebraic multiplicity 1), and the remaining eigenvalues are 0 (with algebraic multiplicity $ n - 1 $).

After multiplying $ J $ by $ x $, the non - zero eigenvalue becomes $ x\cdot n $, and the remaining eigenvalues are still 0. After adding $ (1 - x)I $, each eigenvalue increases $ (1 - x) $.

Therefore, the eigenvalues of matrix $ A $ are:

$ (n-1)x+1 $, with corresponding algebraic multiplicity 1;

$ 1 - x $, with corresponding algebraic multiplicity $ n - 1 $.


\subsection*{(b)} 

The matrix is positive definite $\Leftrightarrow$ all its eigenvalues greater than 0. From (a), we can easy to know $A$ is positive definite $\Leftrightarrow (n-1)x+1 >0 $ and $ 1 - x>0 \Leftrightarrow \frac{1}{1-n}<x<1$.

\section{}

\subsection*{(a)} 

Consider the singular value decomposition of matrix $ X $: $ X = U\Sigma V^H $, where $ U $ and $ V $ are unitary matrices, and $ \Sigma $ is a diagonal matrix whose diagonal entries are the singular values $ \sigma_1, \sigma_2, \ldots, \sigma_n $. Construct the matrix $ Y $ in set $ S $ as $ Y = U\begin{bmatrix}I_n\\0\end{bmatrix}V^H $, where $ \begin{bmatrix}I_n\\0\end{bmatrix} $ is an $ m\times n $ matrix.

$ \forall Y\in S $, $ Y^H Y = I_n $.

We can get $ \|X - Y\|_F=\sqrt{\sum_{i = 1}^n(\sigma_i - 1)^2} $, $ \|I_n - X^H X\|_F = \sqrt{\sum_{i = 1}^n(1 - \sigma_i^2)^2} $.

$$
\sum_{i = 1}^n(\sigma_i - 1)^2\leq\sum_{i = 1}^n(1 - \sigma_i^2)^2
$$

Therefore, $ \sqrt{\sum_{i = 1}^n(\sigma_i - 1)^2}\leq\sqrt{\sum_{i = 1}^n(1 - \sigma_i^2)^2} $, that is, $ \text{dist}(X, S)\leq\|I_n - X^H X\|_F $.

\subsection*{(b)} 

Construct a sequence of matrices $ X_k $, where the first singular value is $ k $ and others are 1. At this time, $ \|I_n - X_k^H X_k\|_F = |1 - k^2| $, and $ \text{dist}(X_k, S)=|k - 1| $. When $ k\to\infty $, the ratio $ \frac{|1 - k^2|}{|k - 1|}=k + 1\to\infty $, which shows that there is no such constant $ C $.

\section{}

\subsection*{(a)} 

\textcolor{red}{The eigenvalues of $ J $ should be $\pm i\sqrt{\sigma_j}$, otherwise $J=\begin{pmatrix}0&A\\-A^H&0\end{pmatrix}$ should be $\begin{pmatrix}0&A\\A^H&0\end{pmatrix}$. Here we modify the proof to show that the eigenvalues of $J$ are $\pm i\sqrt{\sigma_j}$ to ensure the consistency between parts (a) and (b)}.
	
Consider the eigenvalue equation of $ J $: $ J\mathbf{v} = \lambda\mathbf{v} $, where $ \mathbf{v}=\begin{pmatrix}\mathbf{x}\\\mathbf{y}\end{pmatrix} $, $ \mathbf{x}, \mathbf{y}\in\mathbb{C}^n$. Then

$$
J\mathbf{v}=\begin{pmatrix}0&A\\-A^H&0\end{pmatrix}\begin{pmatrix}\mathbf{x}\\\mathbf{y}\end{pmatrix}=\begin{pmatrix}A\mathbf{y}\\-A^H\mathbf{x}\end{pmatrix}=\lambda\begin{pmatrix}\mathbf{x}\\\mathbf{y}\end{pmatrix}
$$

That is,

$$
A\mathbf{y}=\lambda\mathbf{x},\quad(1)
$$
$$
-A^H\mathbf{x}=\lambda\mathbf{y}.\quad(2)
$$

If $ \lambda = 0 $, then $ A\mathbf{y}=\mathbf{0} $ and $ A^H\mathbf{x}=\mathbf{0} $. Since $ A $ is non - singular, we get $ \mathbf{y}=\mathbf{0} $ and $ \mathbf{x}=\mathbf{0} $, that is, $ \mathbf{v}=\mathbf{0} $, a contradiction, so $ \lambda\neq 0 $. Solve $ \mathbf{x}=\lambda^{- 1}A\mathbf{y} $ from (1) and substitute into (2):

$$
-A^H(\lambda^{- 1}A\mathbf{y})=\lambda\mathbf{y}\implies -\lambda^{- 1}A^H A\mathbf{y}=\lambda\mathbf{y}\implies A^H A\mathbf{y}=-\lambda^2\mathbf{y}
$$

Therefore, $ \lambda^2 $ is an eigenvalue of $ A^H A $, that is, $ \lambda^2=- \sigma_j $. Since $ \sigma_j>0 $, we have

$$
\lambda=\pm i\sqrt{\sigma_j}
$$

\subsection*{(b)} 

\textcolor{red}{Failed to prove it}

According to the eigenvalue decomposition, we have:
\begin{align*}
	J &= \begin{pmatrix} U_1 & U_2 \\ V_1 & V_2 \end{pmatrix} \begin{pmatrix} \Sigma & 0 \\ 0 & -\Sigma \end{pmatrix} \begin{pmatrix} U_1^H & V_1^H \\ U_2^H & V_2^H \end{pmatrix}\\
	&= \begin{pmatrix} U_1 & U_2 \\ V_1 & V_2 \end{pmatrix} \begin{pmatrix} \Sigma U_1^H & \Sigma U_2^H \\ -\Sigma V_1^H & -\Sigma V_2^H \end{pmatrix}\\
	&=\begin{pmatrix} U_1\Sigma U_1^H + U_2(-\Sigma V_1^H) & U_1\Sigma U_2^H + U_2(-\Sigma V_2^H) \\ V_1\Sigma U_1^H + V_2(-\Sigma V_1^H) & V_1\Sigma U_2^H + V_2(-\Sigma V_2^H) \end{pmatrix}\\
	&= \begin{pmatrix} 0 & A \\ -A^H & 0 \end{pmatrix}
\end{align*}

So
$$ U_1\Sigma U_1^H + U_2(-\Sigma V_1^H) = 0 $$
$$ U_1\Sigma U_2^H + U_2(-\Sigma V_2^H) = A $$
$$ V_1\Sigma U_1^H + V_2(-\Sigma V_1^H) = -A^H $$
$$ V_1\Sigma U_2^H + V_2(-\Sigma V_2^H) = 0 $$

\section{}

\subsection*{(a)} 

$ \mathbf{e}_i $ is the standard basis vector in $ \mathbb{R}^n $, $ \mathbf{s} = \mathbf{e}_1 + \mathbf{e}_2 + \cdots + \mathbf{e}_n $ is the all - one vector, and $ a $ is a positive number. Let $ \mathbf{v}_i = \mathbf{e}_i - a\mathbf{s} $

For different $ i $ and $ j $
$$
\mathbf{v}_i^T\mathbf{v}_j = (\mathbf{e}_i - a\mathbf{s})^T(\mathbf{e}_j - a\mathbf{s}) = - 2a + a^2n
$$
When $ a\in(0, 2/n) $, the inner product $ - 2a + a^2n < 0 $.


Add the vector $ \mathbf{v}_{n + 1} = -b\mathbf{s} $, where $ b > 0 $.
$$
\mathbf{v}_i^T\mathbf{v}_{n + 1} = (\mathbf{e}_i - a\mathbf{s})^T(-b\mathbf{s}) = b(- 1 + an)
$$
When $ a\in(0, 1/n) $, the inner product $ b(- 1 + an) < 0 $.

In conclusion, When $ a\in(0, 1/n) $, $\mathbf{v}_i^T\mathbf{v}_j <0 \quad i\neq j$

\subsection*{(b)} 

\textcolor{red}{Failed to prove it}

\section{}

Define a linear operator $ T: \mathbb{R}^{m \times n} \to \mathbb{R}^{m \times n} $ as $ T(X) = AX - XB $. Then the original equation is equivalent to $ T(X) = C $. Since $ \mathbb{R}^{m \times n} $ is a finite - dimensional vector space, the following are equivalent for the linear operator $ T $ to have a unique solution for all $ C $: $ T $ is invertible $ \Leftrightarrow $ $ T $ is bijective $ \Leftrightarrow $ $ T $ is injective, that is, the equation $ T(X) = 0 $ has only the zero solution:$AX - XB = 0 \Leftrightarrow AX = XB$ has a unique solution $ X = 0 $.

Therefore, we only need to prove that: $ AX = XB $ has only the zero solution $ \Leftrightarrow $ $ A $ and $ B $ have no common eigenvalues.

$\Rightarrow$

Suppose $ A $ and $ B $ have a common eigenvalue $ \lambda $. Let $ \mathbf{u} $ be an eigenvector of $ A $ belonging to $ \lambda $ (i.e., $ A\mathbf{u} = \lambda \mathbf{u} $, $ \mathbf{u} \neq \mathbf{0} $), and let $ \mathbf{v} $ be a left eigenvector of $ B $ belonging to $ \lambda $ (i.e., $ \mathbf{v}^T B = \lambda \mathbf{v}^T $, $ \mathbf{v} \neq \mathbf{0} $). Construct the matrix $ X = \mathbf{u} \mathbf{v}^T \in \mathbb{R}^{m \times n} $. $ X \neq \mathbf{0} $ because $ \mathbf{u} \neq \mathbf{0} $ and $ \mathbf{v} \neq \mathbf{0} $.
$$
AX = A(\mathbf{u} \mathbf{v}^T) = (A \mathbf{u}) \mathbf{v}^T = (\lambda \mathbf{u}) \mathbf{v}^T = \lambda \mathbf{u} \mathbf{v}^T
$$
$$
XB = (\mathbf{u} \mathbf{v}^T) B = \mathbf{u} (\mathbf{v}^T B) = \mathbf{u} (\lambda \mathbf{v}^T) = \lambda \mathbf{u} \mathbf{v}^T
$$

Thus, $ AX = \lambda \mathbf{u} \mathbf{v}^T = XB $, that is, $ AX = XB $ has a non - zero solution $ X = \mathbf{u} \mathbf{v}^T $.

$\Leftarrow$

\textbf{Lemma}:Suppose $ X $ satisfies $ A X = X B $. $ \forall k \geq 0 $, we have $ A^k X = X B^k $.  

\textbf{Proof}: When $ k = 0 $, $ A^0 X = I X = X $, and $ X B^0 = X B^0 = X I = X $. Thus, $ A^0 X = X B^0 $ holds. 
 
Suppose for some integer $ k \geq 0 $, $ A^k X = X B^k $ holds.  
$$
A^{k + 1} X = A \cdot A^k X = A \cdot (X B^k) = (A X) B^k = (X B) B^k = X B^{k + 1}
$$  

Thus, by mathematical induction, we prove $ A^k X = X B^k $ holds for all integers $ k \geq 0 $  

Let $ f(\lambda) $ be the characteristic polynomial of matrix $ A $. According to the Cayley - Hamilton theorem, $ f(A)=0 $.  

Since $ A $ and $ B $ have no common eigenvalues, all eigenvalues of $ B $ are not roots of $ f(\lambda) $. So $ f(B) $ is an invertible matrix.  

From the lemma, for any polynomial $ p(\lambda) $, we have $ p(A) X = X p(B) $. Let $ p(\lambda)=f(\lambda) $ . Then by the Cayley - Hamilton theorem, $ f(A)=0 $. Thus:  
$$
0 = f(A) X = X f(B)
$$  

Since $ f(B) $ is invertible, from $ X f(B)=0 $, we can obtain:  
$$
X f(B)=0 \implies X = X f(B) f(B)^{-1}=0 \cdot f(B)^{-1}=0
$$  

Thus, when $ A $ and $ B $ have no common eigenvalues, the equation $ A X - X B = 0 $ has only the zero solution:  $X = 0$

Thus, $ AX - XB = C $ has a unique solution for all $ C \in \mathbb{R}^{m \times n} $ $ \Leftrightarrow $ $ A $ and $ B $ have no common eigenvalues.

\section{}
Suppose the tangent line equation of the function $ f(x) $ at $ x = x_0 $ is:$l(x)=ax + b$, where $ a = f'(x_0) $ and $ b = f(x_0)-ax_0 $.

Since $f$ is a convex functions, it satisfies:
$$
f(x_1)\geq f(x_2)+f'(x_2)(x_1 - x_2)
$$
So
$$
f(x)\geq f(x_0)+f'(x_0)(x - x_0)=ax + b
$$

Taking expectations on both sides simultaneously, we have:
$$
E[f(x)]\geq E[ax + b]=aE[x]+b
$$

We take $ x_0 = E[x] $, and correspondingly $ a = f'(x_0) $, $ b = f(x_0)-ax_0 $. Substituting into the above formula at this time, we have:
$$
E[f(x)]\geq aE[x]+b = ax_0 + b = f(x_0)=f(E[x])
$$

\section{}
$f$ is a convex function on $[0,1]$, $\forall x,y \in [0,1]$ and $t\in[0,1]$, we have
$$
f(tx+(1-t)y)\leq tf(x)+(1-t)f(y)
$$

Let $x=0, y=1, t=\frac{1}{2}$, we can get$f(\frac{1}{2})\leq \frac{1}{2}f(0)+\frac{1}{2}f(1)$

$$
f(x) \geq f(0) + \frac{f(1) - f(0)}{1-0} x = f(0) + (f(1) - f(0)) x.
$$
So
$$
\int_{0}^{1} f(x) dx \geq \int_{0}^{1} \left[ f(0) + (f(1) - f(0)) x \right] dx =\frac{1}{2}f(0)+\frac{1}{2}f(1)
$$
So
$$
f(\frac{1}{2}) \leq \int_{0}^{1} f(x) dx
$$

By the property of convex functions, for any $x \in [0, 1]$, we have:
$$
f(x) \leq (1-x) f(0) + x f(1)
$$
So
$$
\int_{0}^{1} f(x) dx \leq \int_{0}^{1} (1-x) f(0) + x f(1)  dx
$$
Calculating the right-hand side integral:
$$
	\int_{0}^{1} (1-x) f(0) + x f(1)  dx =\frac{1}{2}f(0)+\frac{1}{2}f(1)
$$
So
$$
\int_{0}^{1} f(x) \, dx \leq \frac{1}{2} [f(0) + f(1)]
$$
Therefore
$$
f(\frac{1}{2}) \leq \int_{0}^{1} f(x) dx \leq \frac{1}{2} [f(0) + f(1)]
$$

\section{}

Consider two independent and identically distributed random variables $ X $ and $ Y $, both of which have the same distribution as $ X $. Since $ f $ and $ g $ are increasing functions, for any real numbers $ x $ and $ y $, we have:

When $ x \geq y $, $ f(x) \geq f(y) $ and $ g(x) \geq g(y) $, so $ (f(x) - f(y))(g(x) - g(y)) \geq 0 $.
When $ x < y $, $ f(x) \leq f(y) $ and $ g(x) \leq g(y) $, so $ (f(x) - f(y))(g(x) - g(y)) \geq 0 $.

Therefore, for all $ x, y $, we have:
$$
(f(x) - f(y))(g(x) - g(y)) \geq 0
$$

Take the expectation, we can obtain:
$$
\mathbb{E}[(f(X) - f(Y))(g(X) - g(Y))] \geq 0
$$

Expand the left - hand side, that is, 
$\mathbb{E}[f(X)g(X)-f(X)g(Y)-f(Y)g(X)+f(Y)g(Y)]$

Since $ X $ and $ Y $ are independent and and identical distribution:
$$
\mathbb{E}[f(X)g(Y)]=\mathbb{E}[f(X)]\mathbb{E}[g(Y)]=\mathbb{E}[f(X)]\mathbb{E}[g(X)]
$$
$$\mathbb{E}[g(Y)] = \mathbb{E}[g(X)]
$$
$$
\mathbb{E}[f(Y)g(X)]=\mathbb{E}[f(Y)]\mathbb{E}[g(X)]=\mathbb{E}[f(X)]\mathbb{E}[g(X)]
$$
$$
\mathbb{E}[f(Y)g(Y)]=\mathbb{E}[f(X)g(X)]
$$ 

Substitute these in:
$$\mathbb{E}[f(X)g(X)]-\mathbb{E}[f(X)]\mathbb{E}[g(X)]-\mathbb{E}[f(X)]\mathbb{E}[g(X)]+\mathbb{E}[f(X)g(X)] \geq 0$$

That is:
$$
2\mathbb{E}[f(X)g(X)] - 2\mathbb{E}[f(X)]\mathbb{E}[g(X)] \geq 0
$$

Therefore:
$$
\mathbb{E}[f(X)g(X)] \geq \mathbb{E}[f(X)]\mathbb{E}[g(X)]
$$

\section{}

\subsection*{Proof 1} 

From the Chebyshev's inequality we can obtain that for two sequences $\{a_k\}$ and $\{b_k\}$ that are monotonic in the same direction, we have:

$$
\frac{1}{n + 1}\sum_{k = 0}^{n}a_kb_k\geq\left(\frac{1}{n + 1}\sum_{k = 0}^{n}a_k\right)\left(\frac{1}{n + 1}\sum_{k = 0}^{n}b_k\right)
$$
So
$$
\sum_{k = 0}^{n}a_kb_k\geq\frac{1}{n + 1}\left(\sum_{k = 0}^{n}a_k\right)\left(\sum_{k = 0}^{n}b_k\right)
$$

Consider $\{a_k\}$ and $\{b_{n - k}\}$. Since  $\{b_k\}$ and $\{a_k\}$ are monotonic in the same direction, if $\{b_k\}$ is increasing, then $\{b_{n - k}\}$ is decreasing. At this time, $\{a_k\}$ and $\{b_{n - k}\}$ are monotonic in opposite directions. According to Chebyshev's inequality, sequences that are monotonic in opposite directions satisfy:

$$
\frac{1}{n + 1}\sum_{k = 0}^{n}a_kb_{n - k}\leq\left(\frac{1}{n + 1}\sum_{k = 0}^{n}a_k\right)\left(\frac{1}{n + 1}\sum_{k = 0}^{n}b_{n - k}\right)
$$

Since $\sum_{k = 0}^{n}b_{n - k}=\sum_{k = 0}^{n}b_k$, the right - hand side becomes:

$$\left(\frac{1}{n + 1}\sum_{k = 0}^{n}a_k\right)\left(\frac{1}{n + 1}\sum_{k = 0}^{n}b_k\right)
$$
So
$$\sum_{k = 0}^{n}a_kb_{n - k}\leq\frac{1}{n + 1}\left(\sum_{k = 0}^{n}a_k\right)\left(\sum_{k = 0}^{n}b_k\right)$$

\section{}

\textcolor{red}{Failed to prove it}

\subsection*{Consider the Converse Proposition} 

The converse proposition is: If the sequence $ \{x_k\} $ converges, then $ \sum_{|x_k| > \epsilon} |x_k - x_{k + 1}| < \infty $ holds for all $ \epsilon > 0 $.

Suppose $ \{x_k\} $ converges to $ a $. Then for any $ \epsilon > 0 $, there exists $ N>0 $ such that for all $ k > N $, $ |x_k - a| < \epsilon/2 $. Therefore, for $ k > N $, we have:

$$
|x_k - x_{k + 1}| \leq |x_k - a| + |a - x_{k + 1}| < \epsilon
$$

From the condition, we can get $|x_k - x_{k + 1}| < \epsilon$,  since $ \{x_k\} $ converges, $\exists N_1>0 $ such that when $k>N$, we have $x_k<\epsilon$. Therefore , $ \sum_{|x_k| > \epsilon} |x_k - x_{k + 1}| < \infty $ has only finitely many terms. So $ \sum_{|x_k| > \epsilon} |x_k - x_{k + 1}| < \infty $

\section{}
\textcolor{red}{Failed to prove it}

\section{}

\subsection*{(a)} 

Consider the function $ f(\mathbf{x}) = \|T(\mathbf{x}) - \mathbf{x}\| $. Since $ T $ is continuous, $ f(\mathbf{x}) $ is continuous on the compact set $ X $. $ f $ attains its minimum value on $ X $. Let the minimum value be attained at $ \mathbf{x}^* \in X $, i.e., $ f(\mathbf{x}^*) = d \geq 0 $.  

If $ d = 0 $, then $ \mathbf{x}^* $ is a fixed point.  

If $ d > 0 $, then $ \mathbf{x}^* \neq T(\mathbf{x}^*) $. Consider $ T(\mathbf{x}^*) \in X $. According to the problem's condition, $ \|T(T(\mathbf{x}^*)) - T(\mathbf{x}^*)\| < \|T(\mathbf{x}^*) - \mathbf{x}^*\| = d $, i.e., $ f(T(\mathbf{x}^*)) < d $, which contradicts the fact that $ d $ is the minimum value. Therefore, $ d $ must be 0, meaning a fixed point exists.  

Suppose there exist two distinct fixed points $ \mathbf{x}^* $ and $ \mathbf{y}^* $, i.e., $ T(\mathbf{x}^*) = \mathbf{x}^* $ and $ T(\mathbf{y}^*) = \mathbf{y}^* $. According to the problem's condition, when $ \mathbf{x} \neq \mathbf{y} $, $ \|T(\mathbf{x}) - T(\mathbf{y})\| < \|\mathbf{x} - \mathbf{y}\| $. But $ \|T(\mathbf{x}^*) - T(\mathbf{y}^*)\| = \|\mathbf{x}^* - \mathbf{y}^*\| $, which contradicts the condition. Therefore, the fixed point must be unique.  

In conclusion, $ T $ has exactly one fixed point on $ X $.  

\subsection*{(b)} 
Since $T$ satisfies $\|T(x) - T(y)\| < \|x - y\|$, we have:
$$
\|x_{k + 1} - x_k\| = \|T(x_k) - T(x_{k - 1})\| < \|x_k - x_{k - 1}\|
$$

This shows that the sequence $\{\|x_{k + 1} - x_k\|\}$ is a decreasing sequence of positive numbers and converges to some limit $a \geq 0$.

Suppose $a > 0$. Then for any $\epsilon > 0$, there exists $N_1$ such that when $k > N_1$, $\|x_{k + 1} - x_k\| < a + \epsilon$. Since $\{\|x_{k + 1} - x_k\|\}$ is decreasing, $a$ must be $0$, that is:
$$
\lim_{k \to \infty} \|x_{k + 1} - x_k\| = 0
$$

Since $\|x_{k + 1} - x_k\|$ is decreasing and converges to $0$, according to the Monotone Convergence Theorem, the series $\sum_{k = 0}^{\infty} \|x_{k + 1} - x_k\|$ converges.

For any $\epsilon > 0$, there exists $N_2$ such that when $n > N_2$
$$
\sum_{k = n}^{\infty} \|x_{k + 1} - x_k\| < \epsilon
$$

For any $m > n > N_2$
$$\|x_m - x_n\| \leq \sum_{k = n}^{m - 1} \|x_{k + 1} - x_k\| < \sum_{k = n}^{\infty} \|x_{k + 1} - x_k\| < \epsilon$$

This shows that the sequence $\{x_k\}$ is a Cauchy sequence.

Since $ X $ is compact, the Cauchy sequence $ \{x_k\} $ must converge to some point $ x^* $ in $ X $, that is, $ \lim_{k \to \infty} x_k = x^* $.

Since $ T $ is continuous, we have:
$$
T(x^*) = T\left( \lim_{k \to \infty} x_k \right) = \lim_{k \to \infty} T(x_k) = \lim_{k \to \infty} x_{k + 1} = x^*
$$

From the result of problem (a), the fixed point of $ T $ on $ X $ is unique. Therefore, no matter how the initial point $ x_0 $ is chosen, the iterative sequence will converge to this unique fixed point $ x^* $.

\section{}
$f: [0,1] \to [0,1]$ is a continuous function. The sequence $ x_k $ is bounded, and by the Bolzano-Weierstrass theorem, we can know that there is a convergent subsequence $ x_{k_j} $.

Let $ x_{k_j} \rightarrow a \in [0,1]$. Since $f$ is a continuous function, $ x_{k_j+1}=f(x_{k_j})\rightarrow a$. Apply $x_k-x_{k+1}\rightarrow 0$ to the subsequence, we can get $f(a)=a$, $a$ is a fixed point of $f$

Assume that both $ a $ and $ b \quad (a \neq b)$ are limit points of the sequence $\{x_n\}$. $a$ and $b$ are fixed points of $f$. Let $|a-b|=d$, and $\exists N>0$ such that when $k>N$, there is $x_k-x_{k+1}<\frac{d}{3}$

If for some $k>N$, $|x_k-a| <\frac{d}{3} $

$$
|x_{k+1}-a|\leq |x_{k+1}-x_k|+|x_k-a|<\frac{2d}{3}
$$
$$
|x_{k+1}-b|\geq|a-b|- |x_{k+1}-a|>\frac{d}{3}
$$

For $k>N$, we have$|x_k - a| < d = |a - b|, \quad |x_k - b| > \frac{d}{3}$. 

If the sequence $ x_k $ converges to $ b $, then there exists a sufficiently large $ k $ such that $ |x_k - b| < d/3 $, which contradicts $ |x_k - b| > \frac{d}{3} $. Therefore, $ x_k $ have only one limit point. Sequence $x_k$ convergence

\section{}
Since $f$ is a twice differentiable function, ee can perform a Taylor expansion to the second derivative term for $f$ at $x=0$ and $x=1$, respectively.

Taylor unfolds at $x=0$:
$$
f(x)=f(0)+f'(0)+\frac{f''(\eta_1)}{2} x^2=f(0)+\frac{f''(\eta_1)}{2} x^2, \quad \eta_1\in (0,x)
$$

Taylor unfolds at $x=1$:
$$
f(x)=f(1)+f'(1)+\frac{f''(\eta_2)}{2} x^2=f(1)+\frac{f''(\eta_2)}{2} x^2, \quad \eta_2\in (x,1)
$$

Substitute $x = \frac{1}{2}$ 

$$
f( \frac{1)}{2}=f(0)+\frac{f''(\eta_1)}{8} 
$$ 
$$
f( \frac{1}{2})=f(1)+\frac{f''(\eta_2)}{8} 
$$

We can have
$$
|f''(\eta_2)-f''(\eta_1)|=8|f(0)-f(1)|
$$

Substitute $x=1$
$$
f(1)=f(0)+\frac{f''(\eta_3)}{2} 
$$

We can have
$$
|f(1)-f(0)|=|\frac{f''(\eta_3)}{2} |
$$

Since Darboux's theorem, $f''$ on $[\eta_1,\eta_2]$ can take all the values between $f''(\eta_1)$ and $f''(\eta_2)$.

If $4|f(0)-f(1)|$ is between $f''(\eta_1)$ and $f''(\eta_2)$, $\exists \xi \in (\eta_1,\eta_2)$, such that $f''(\xi )=4|f(0)-f(1)|$.

If  $-4|f(0)-f(1)|$ is between $f''(\eta_1)$ and $f''(\eta_2)$, we can come to the same conclusion.

If $f''(\eta_1) \geq 4|f(0)-f(1)|$ and $f''(\eta_2) \geq 4|f(0)-f(1)|$, $f''(\eta_3)=2|f(0)-f(1)| < 4|f(0)-f(1)|$, so $\exists \xi \in (\eta_3,\eta_1)$, such that $f''(\xi )=4|f(0)-f(1)|$.

If $f''(\eta_1) \leq -4|f(0)-f(1)|$ and $f''(\eta_2) \leq -4|f(0)-f(1)|$, we can come to the same conclusion.

\section{}

Let $ m = \min_{x \in [0,1]} f(x) $. Suppose $ m > 2 $. Then for any $ x \in [0,1] $, we have $ f(x) \geq m > 2 $. According to the given condition, the integral inequality is:

$$\int_{0}^{x} [f(t)]^2 dt \leq f(x)$$

Since $ f(t) \geq m $, the lower bound of the integral is:

$$\int_{0}^{x} m^2 dt = m^2 x \leq f(x)$$

Let $ x = 1 $, we have $ m^2 \leq f(1) \geq m $, that is, $ m^2 \leq m $, which contradicts $ m > 2 $. Therefore, the assumption does not hold, so $ m \leq 2 $.


\section{}

\textcolor{red}{Failed to prove it}

\section{}

Consider the Singular Value Decomposition of $ A $:
$$
A = U\Sigma V^T
$$

where $ U $ and $ V $ are orthogonal matrices, and $ \Sigma $ is a diagonal matrix whose diagonal entries are the singular values of $ A $, $ \sigma_1 \geq \sigma_2 \geq \cdots \geq \sigma_n \geq 0 $.

Let $ \mathbf{v}_n $ be the right singular vector of $ A $ corresponding to the minimum singular value $ \sigma_n $. According to the properties of singular value decomposition, we have:
$$
A\mathbf{v}_n = \sigma_n \mathbf{u}_n
$$

where $ \mathbf{u}_n $ is the left singular vector and $ \|\mathbf{u}_n\|_2 = 1 $.

Take $ \mathbf{x} = \mathbf{v}_n $. Obviously, $ \|\mathbf{x}\|_2 = 1 $.

For the vector $ A\mathbf{x} $, its infinity norm satisfies:
$$
\|A\mathbf{x}\|_\infty \leq \|A\mathbf{x}\|_2
$$

From the singular value decomposition, we know that:
$$
\|A\mathbf{x}\|_2 = \sigma_n
$$
  
The Frobenius norm of $ A $ is:
$$
\|A\|_F = \sqrt{\sigma_1^2 + \sigma_2^2 + \cdots + \sigma_n^2}
$$

Since $ \sigma_1 \geq \sigma_2 \geq \cdots \geq \sigma_n \geq 0 $
$$
\sigma_n \leq \sqrt{\frac{\sigma_1^2 + \sigma_2^2 + \cdots + \sigma_n^2}{n}} = \frac{\|A\|_F}{\sqrt{n}}
$$

We can know that
$$
\|A\mathbf{x}\|_\infty \leq \|A\mathbf{x}\|_2 = \sigma_n
$$
$$
\sigma_n \leq \frac{\|A\|_F}{\sqrt{n}} \leq \frac{\|A\|_F}{n}
$$
So
$$
\|A\mathbf{x}\|_\infty \leq \frac{\|A\|_F}{n}
$$

where $ \mathbf{x} = \mathbf{v}_n $ is a unit vector.

In conclusion, for any $ A \in \mathbb{R}^{n \times n} $, there exists a unit vector $ \mathbf{x} $ such that:
$$
\min_{\|\mathbf{x}\|_2 = 1} \|A\mathbf{x}\|_\infty \leq \frac{1}{n}\|A\|_F
$$

\section{}

\textcolor{red}{Failed to prove it}


adsds 

\end{document}